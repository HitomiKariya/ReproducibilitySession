%% loading data to do basic R

\documentclass[11pt]{article}


\title{My first replicable Paper}
\author{
        MyFirstName MyLastName\\
        Evans School of Public Policy and Governance\\
        University of Washington\\
        Seattle, WA 98115, \underline{United States}\\
        \texttt{greatguy@uw.edu}
}
\date{\today}


\usepackage{Sweave}
\begin{document}
\Sconcordance{concordance:PaperInR_2.tex:PaperInR_2.Rnw:%
1 16 1 1 0 30 1 1 3 2 0 1 5 4 0 1 1 4 0 1 3 5 1 1 2 1 0 1 1 7 0 2 2 4 0 %
1 2 7 1 1 2 14 0 1 2 1 1 1 2 4 0 1 2 10 1 1 2 4 0 1 2 6 1 1 2 4 0 1 2 5 %
1}


\maketitle 

\begin{abstract}
This is an example on how to make a reproducible paper. We are using R from Rstudio, creating an RSweave document. This is a nice start to create a nice paper and get an A+. The next sections will show the steps taken.
\end{abstract}


\section{Introduction}\label{intro}  

This is my intro to my great paper, I will explain the cool things I can do with my new `computational thinking' powers combined with some Latex. This is my intro to my great paper, I will explain the cool things I can do with my new `computational thinking' powers combined with some Latex. This is my intro to my great paper, I will explain the cool things I can do with my new `computational thinking' powers combined with some Latex. This is my intro to my great paper, I will explain the cool things I can do with my new `computational thinking' powers combined with some Latex.


This is my nice intro to my great paper, 
I will explain the cool things 
I can do with my new `computational thinking' 
powers
combined with some Latex.


\section{Exploring Data}\label{explo}


Sections may use a label\footnote{In fact, you can have a label wherever you think a future reference to that content might be needed.}. This label is needed for referencing. For example the next section has label \emph{datas}, so you can reference it by writing: As we see in section \ref{catexplo}.


%%%%%% code for loading here

% this is a CODE CHUNK
\begin{Schunk}
\begin{Sinput}
> # collecting
> fileLink="https://github.com/HitomiKariya/MyFirstRepo/raw/master/dataidx.RDS"
> # you need to copy the link from "download" button on the repo
> # right click on the "download" -> "Copy the link"
> # if you just copy the link on the google window, it won't work
> 
> MyFile=url(fileLink)
> dataidx=readRDS(MyFile)
> 
\end{Sinput}
\end{Schunk}

\subsection{Exploring Categorical Data}\label{catexplo}

Here, I continue doing this nice work, I hope you like it and read it. It has been a very hard work.Here, I continue doing this nice work, I hope you like it and read it. It has been a very hard work.Here, I continue doing this nice work, I hope you like it and read it. It has been a very hard work.Here, I continue doing this nice work, I hope you like it and read it. It has been a very hard work.Here, I continue doing this nice work, I hope you like it and read it. It has been a very hard work.Here, I continue doing this nice work, I hope you like it and read it. It has been a very hard work.Here, I continue doing this nice work, I hope you like it and read it. It has been a very hard work.Here, I continue doing this nice work, I hope you like it and read it. It has been a very hard work.Here, I continue doing this nice work, I hope you like it and read it. It has been a very hard work.

%%%%%% code for exploring here
\begin{Schunk}
\begin{Sinput}
> tableONI=table(dataidx$ONIpolitical)
> tableONI
\end{Sinput}
\begin{Soutput}
 nd per sub sel  ne 
  2   8   4  21  41 
\end{Soutput}
\end{Schunk}

\begin{Schunk}
\begin{Sinput}
> barplot(tableONI)
\end{Sinput}
\end{Schunk}



\subsection{Exploring Numerical Data}\label{numexplo}

Here, I continue doing this nice work, I hope you like it and read it. It has been a very hard work.Here, I continue doing this nice work, I hope you like it and read it. It has been a very hard work.Here, I continue doing this nice work, I hope you like it and read it. It has been a very hard work.Here, I continue doing this nice work, I hope you like it and read it. It has been a very hard work.Here, I continue doing this nice work, I hope you like it and read it. It has been a very hard work.Here, I continue doing this nice work, I hope you like it and read it. It has been a very hard work.Here, I continue doing this nice work, I hope you like it and read it. It has been a very hard work.Here, I continue doing this nice work, I hope you like it and read it. It has been a very hard work.Here, I continue doing this nice work, I hope you like it and read it. It has been a very hard work.

%%%%%% code for exploring here
\begin{Schunk}
\begin{Sinput}
> summary(dataidx[,c(3,4)])
\end{Sinput}
\begin{Soutput}
      FHF             RWB       
 Min.   :10.00   Min.   : 6.38  
 1st Qu.:25.25   1st Qu.:23.60  
 Median :49.00   Median :28.72  
 Mean   :47.24   Mean   :32.40  
 3rd Qu.:63.00   3rd Qu.:38.50  
 Max.   :97.00   Max.   :84.83  
 NA's   :5       NA's   :23     
\end{Soutput}
\end{Schunk}


\begin{Schunk}
\begin{Sinput}
> boxplot(dataidx[,c(3,4)])
\end{Sinput}
\end{Schunk}

Boxplots were introduced by Tuckey (Tukey, John W (1977). Exploratory Data Analysis. Addison-Wesley.)

\section{Looking for Relationships}\label{bivar}


Here, I continue doing this nice work, I hope you like it and read it. It has been a very hard work.Here, I continue doing this nice work, I hope you like it and read it. It has been a very hard work.Here, I continue doing this nice work, I hope you like it and read it. It has been a very hard work.Here, I continue doing this nice work, I hope you like it and read it. It has been a very hard work.Here, I continue doing this nice work, I hope you like it and read it. It has been a very hard work.Here, I continue doing this nice work, I hope you like it and read it. It has been a very hard work.Here, I continue doing this nice work, I hope you like it and read it. It has been a very hard work.Here, I continue doing this nice work, I hope you like it and read it. It has been a very hard work.Here, I continue doing this nice work, I hope you like it and read it. It has been a very hard work.

\subsection{Numerical and  Categorical}\label{binumcat}

%%%%%% code for exploring here
\begin{Schunk}
\begin{Sinput}
> boxplot(dataidx$FHF~dataidx$Region)
\end{Sinput}
\end{Schunk}

Here, I continue doing this nice work, I hope you like it and read it. It has been a very hard work.Here, I continue doing this nice work, I hope you like it and read it. It has been a very hard work.Here, I continue doing this nice work, I hope you like it and read it. It has been a very hard work.Here, I continue doing this nice work, I hope you like it and read it. It has been a very hard work.Here, I continue doing this nice work, I hope you like it and read it. It has been a very hard work.Here, I continue doing this nice work, I hope you like it and read it. It has been a very hard work.Here, I continue doing this nice work, I hope you like it and read it. It has been a very hard work.Here, I continue doing this nice work, I hope you like it and read it. It has been a very hard work.Here, I continue doing this nice work, I hope you like it and read it. It has been a very hard work.

\subsection{Numerical and Numerical}\label{binumnum}

Here, I continue doing this nice work, I hope you like it and read it. It has been a very hard work.Here, I continue doing this nice work, I hope you like it and read it. It has been a very hard work.Here, I continue doing this nice work, I hope you like it and read it. It has been a very hard work.Here, I continue doing this nice work, I hope you like it and read it. It has been a very hard work.Here, I continue doing this nice work, I hope you like it and read it. It has been a very hard work.Here, I continue doing this nice work, I hope you like it and read it. It has been a very hard work.Here, I continue doing this nice work, I hope you like it and read it. It has been a very hard work.Here, I continue doing this nice work, I hope you like it and read it. It has been a very hard work.Here, I continue doing this nice work, I hope you like it and read it. It has been a very hard work.
%%%%%% code for exploring here
\begin{Schunk}
\begin{Sinput}
> plot(dataidx$FHF~dataidx$RWB)
\end{Sinput}
\end{Schunk}

The scatter plot is thought to be invented by  John Frederick W. Herschel according to this link: https://qz.com/1235712/the-origins-of-the-scatter-plot-data-visualizations-greatest-invention/



\end{document}
